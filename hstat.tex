\documentclass[10pt,letterpaper,oneeqnum,final]{siamltex}
%\documentclass[10pt,letterpaper]{amsart}
\usepackage{amsmath,amssymb,bm}
\usepackage{graphicx,url,verbatim}
\usepackage{microtype}
\usepackage{ucs}
\usepackage[utf8x]{inputenc}

\setlength{\textwidth}{27pc}
\setlength{\textheight}{43pc}
%\linespread{1.6}

\newcommand{\EE}{\mathcal E}
\newcommand{\KK}{\mathsf K}
\newcommand{\PP}{\mathsf P}
\newcommand{\VV}{\bm V}
\newcommand{\R}{\mathbb R}
\newcommand{\bigO}{{\mathcal{O}}}
\newcommand{\abs}[1]{{\left\lvert #1 \right\rvert}}
\newcommand{\norm}[1]{{\left\lVert #1 \right\rVert}}
\newcommand{\tcolon}{{ : }}
\newcommand{\ip}[2]{{\left\langle #1, #2 \right\rangle}}

\DeclareMathOperator{\sspan}{span}

\title{Achieving textbook multigrid efficiency for hydrostatic ice sheet flow}

\author{Jed Brown\thanks{Versuchsanstalt f\"ur Wasserbau, Hydrologie und Glaziologie (VAW), ETH Z\"urich, Z\"urich,
    Switzerland, \texttt{brown@vaw.baug.ethz.ch}}
  \and Barry Smith\thanks{Mathematics and Computer Science Division, Argonne National Laboratory, Argonne, IL 60439, \texttt{bsmith@mcs.anl.gov}}
  %\and Dmitry Karpeev\thanks{Mathematics and Computer Science Division, Argonne National Laboratory, Argonne, IL 60439, \texttt{karpeev@mcs.anl.gov}}
}

\begin{document}
\maketitle
\begin{abstract}
  The hydrostatic equations for ice sheet flow offer improved fidelity compared to the shallow ice
  approximation and shelfy stream approximation popular in today's ice sheet models, but
  it presents a serious bottleneck because it requires the solution of a 3D nonlinear system, as
  opposed to the 2D system present in SSA.  This system is posed on high-aspect domains with strong
  anisotropy and variation in coefficients, making it very expensive to solve using the methods
  currently employed.  This paper presents a multigrid Newton-Krylov solver for the hydrostatic
  equations that demonstrates textbook multigrid efficiency (an order of magnitude reduction in
  residual per iteration and solution of the fine-level system in a small multiple of the cost of a
  residual evaluation).
\end{abstract}
\begin{keywords}
  hydrostatic, ice sheet, Newton-Krylov, multigrid, preconditioning
\end{keywords}
\thispagestyle{plain}
\pagestyle{myheadings}
\markboth{J. BROWN}{Textbook multigrid efficiency for hydrostatic ice sheet}

\section{Introduction}
The dynamic response of ice streams and outlet glaciers is poorly represented using the shallowness
assumptions inherent in the present generation of ice sheet models.  Accurate simulation of this
response is crucial for prediction of sea level rise and the inability of available models, based on
the shallow ice approximation (SIA)~\cite{hutter1983tgm} and shallow stream approximation
(SSA)~\cite{morland1987unconfined,weis1999theory}, to simulate these processes was cited as a major
deficiency in the IPCC AR4~\cite{ipcc2007ar4-syr}.

The hydrostatic equations were introduced by \cite{blatter1995vas} as a model of intermediate
complexity between the full non-Newtonian Stokes system and the integrated SIA and SSA models.  A
more precise analysis, including the limiting cases of fast and slow sliding, was given in
\cite{schoof2010thin}.  Well-posedness was proven in \cite{colinge1999strongly} and approximation
properties of finite element methods were analyzed in
\cite{glowinski2003approximation,chow2004finite}.  The hydrostatic equations were used for transient
simulation in \cite{pattyn2002tgr} and in 3D in \cite{pattyn2003ntd}, as well as subsequent work.
Several models of this form were compared in \cite{pattyn2008beh}.

However, the use of hydrostatic equations in current models has been limited due to the cost of solving
the 3D nonlinear system for velocity.  This cost comes from both slow convergence on the
nonlinearities (rheology and slip), and expensive linear solves using standard preconditioners such
as incomplete factorization and 1-level domain decomposition.  The poor linear solve performance is
attributable to the strong anisotropy and heterogeneity imposed by the rheology and geometry.

In the present work, we present a Newton-Krylov solver that demonstrates textbook multigrid
efficiency (TME), characterized by convergence in a small multiple of the cost of a single
fine-level residual evaluation, and typically involving an order of magnitude reduction in residual
per multigrid (V or F) cycle.  The scheme converges quadratically on the nonlinearities, is rapidly
globalized using grid sequencing, is very robust to parameters and geometry, coarsens rapidly in
almost all cases, and exhibits excellent parallel scalability.  Our code is freely available as part
of the Portable Extensible Toolkit for Scientific computing (PETSc)~\cite{petsc-web-page}.

Section~\ref{sec:equations} presents the equations and discretization, Section~\ref{sec:solver}
describes the solver, and Section~\ref{sec:examples} demonstrates the performance and scalability
with numerical examples.

\section{Equations and discretization}\label{sec:equations}
The hydrostatic equations are obtained from the non-Newtonian Stokes equations in the limit where
horizontal derivatives of vertical velocity are small.  Neglecting these terms allows
incompressibility to be enforced implicitly by eliminating pressure and vertical velocity, leaving a
system involving only horizontal components of velocity.  See \cite{schoof2010thin} for a rigorous
derivation and asymptotic analysis.

Consider an ice domain $\Omega \subset \R^3$ lying between the bed $b \in C^1(\R^2)$ and surface $s
\in C^1(\R^2)$, with thickness $H = s-b$ bounded below by a positive constant\footnote{This singular
  limit is important in the case of grounded margins, but the present work does not pursue it.}  The
velocity $\bm u = (u,v) \in \bm V = \bm H^1(\Omega)$ satisfies conservation of momentum which,
omitting inertial terms as is standard for ice sheets, is given by
\begin{align}\label{eq:momentum}
  - \nabla\cdot \left[ \eta
  \begin{pmatrix}
    4 u_x + 2 v_y & u_y + v_x & u_z \\
    u_y + v_x & 2 u_x + 4 v_y & v_z
  \end{pmatrix} \right] + \rho g \nabla s & = 0
\end{align}
where
\begin{align}\label{eq:viscosity}
  \eta(\gamma) = \frac B 2 (\epsilon^2 + \gamma)^{\frac{1-n}{2n}}
\end{align}
is the nonlinear effective viscosity with regularization $\epsilon$ and
\begin{align*}
  \gamma = u_x^2 + v_y^2 + u_xv_y + \frac 1 4 (u_y+v_x)^2 + \frac 1 4 u_z^2 + \frac 1 4 v_z^2
\end{align*}
is the second invariant.  Ice sheet models typically take $n=3$ as the power law exponent.  Equation
\eqref{eq:momentum} is subject to natural boundary conditions at the free surface and either no-slip
$\bm u = 0$ or power-law slip with friction parameter
\begin{align*}
  \beta^2(\gamma_b) = \beta_0^2 (\epsilon_b^2 + \gamma_b)^{\frac{m-1}{2}}
\end{align*}
where $\gamma_b = \frac 1 2 (u^2 + v^2)$, $\epsilon_b$ is regularization, and $m \in (0,1]$ is the
exponent that produces Navier slip for $m=1$, Weertman~\cite{weertman1957sliding} sliding for
$m=1/n$, and Coulomb slip as $m \to 0$.

To discretize this system with a finite element method, we introduce the test functions $\bm \phi
\in \VV$ and integrate by parts to produce the weak form: find $\bm u \in \VV$ such that
\begin{align}\label{eq:weak}
  \int_\Omega \nabla\bm\phi \tcolon \eta
  \begin{pmatrix}
    4 u_x + 2 v_y & u_y + v_x & u_z \\
    u_y + v_x & 2 u_x + 4 v_y & v_z
  \end{pmatrix} + \bm\phi \cdot\rho g \nabla s
  + \int_{\Gamma_{\text{bed}}} \bm \phi \cdot \beta^2(\abs{\bm u}^2/2) \bm u = 0
\end{align}
for all $\bm \phi \in \VV$ where $\Gamma_{\text{bed}}$ is the slip portion of the bed.

Equation \eqref{eq:weak} was discretized on a structured hexahedral grid using $Q_1$ finite elements
and standard 8-point Gauss quadrature.  Length, time, and mass units were chosen so that thickness
and velocities are $\bigO(1)$ and $\rho g = 1$.  See Section~\ref{ssec:dirichlet} for details on the
enforcement of Dirichlet boundary conditions.

\section{Solver and implementation}\label{sec:solver}
We begin by writing the discretization of \eqref{eq:weak} as an algebraic system $F(U) = 0$ with
Jacobian $J(U)$.  This nonlinear system is solved with a Newton iteration which requires an
(approximate) solution $\delta U$ of
\begin{align}\label{eq:newton-step}
  J(U)\delta U = -F(U) .
\end{align}
Newton methods are quadratically convergent in the terminal phase, but may converge very slowly or
not at all in the early phases.  Most applications of the present solver are in a time-stepping code
where the initial iterates start within the region of quadratic convergence, thus globalization
would rarely be a concern, but since a good initial iterate is not available in the present tests,
we use grid sequencing (solving the problem on a sequence of coarser grids) to produce an initial
iterate on the fine grid.

The Newton step \eqref{eq:newton-step} is solved by a Krylov method such as GMRES, for which the
iteration count is highly dependent on quality of the preconditioner.  Since $J(U)$ is symmetric
positive definite, methods such as conjugate gradients could be used, however this work always uses
GMRES because it allows the use of nonsymmetric preconditioners such as F-cycle multigrid, and the
iteration counts are always kept very low so that storage and restarts are not an issue.  As an SPD
system, a wide variety of preconditioners are available, however, the large viscosity contrasts,
strong anisotropy, and significance of shear terms cause most preconditioners to perform poorly.
The rest of this section describes the methods used to produce a scalable algorithm in spite of
these difficulties.

\subsection{Anisotropy}
The aspect ratio of outlet glaciers (the regions of ice sheets with greatest physical interest)
range from $\bigO(1)$ to over $100$.  The nonlinear constitutive relation \eqref{eq:viscosity}
produces several orders of magnitude variation in viscosity (usually with fastest variation in the
vertical) and the Newton linearization of \eqref{eq:weak} produces additional anisotropy,
effectively collapsing the conductivity tensor in the direction of the velocity gradient.

For systems with a priori known anisotropy, semi-coarsening has been successful for attaining
satisfactory multigrid performance even with weak smoothers like SOR, but semi-coarsening is
unattractive for two reasons.  Firstly, semi-coarsening in the vertical direction would necessitate
many levels because it only reduces the problem size by a factor of 2 on each coarsening (instead of
8 for isotropic coarsening), and the fully-coarsened problem would still be far too large for a
direct solver, necessitating further coarsening in the horizontal direction.  The presence of many
levels leads to more synchronization in parallel which is detrimental to scalability and makes
performance more sensitive to network latency.  Second, large viscosity contrasts and anisotropy
unaligned with the grid arise when friction parameter $\beta^2$ is not smooth, which is the standard
case when studying the migration of ice stream margins as the bed transitions from frozen (no-slip
or very high friction) to temperate and very slippery depending on subglacial hydrology.

In order to coarsen isotropically even on high-aspect domains, we order the unknowns so that columns
are contiguous with a natural block size of 2
(i.e. $\{u_{i,j,k},v_{i,j,k},u_{i,j,k+1},v_{i,j,k+1},\dotsc\}$ where $k$ is the index which is
increasing in the vertical direction) and not decomposed in parallel.  This decomposition is
reasonable since the number of vertical levels used in simulations is typically between 10 and 100,
and convenient since it is compatible with decompositions used by other climate model components.

With this ordering, zero-fill incomplete factorization effectively performs an exact solve of the
column since all the neglected fill (relative to an exact factorization) is caused by the coupling
with adjacent columns.  One might expect that with this smoother, a coarse level would not even be
required in the high-aspect frozen bed (no slip) case, since a lubrication approximation (SIA) is
valid in this regime, and thus velocities can be determined locally from driving stress.  However
this is not the case since certain shear modes are invisible to the preconditioner %TODO

We have observed an interesting weakness of incomplete Cholesky factorizations that make it unusable
as a stand-alone preconditioner, and 

\subsection{Dirichlet boundary conditions}\label{ssec:dirichlet}
Multigrid is often sensitive to the enforcement of boundary conditions.  Ideally, Dirichlet
conditions would be completely removed from the solution space, but doing so complicates grid
management on structured grids, so instead we leave these degrees of freedom in the system but
decouple them from the other equations.  During residual evaluation in the finite element context,
this corresponds to evaluating integrals with the Dirichlet condition satisfied exactly, and setting
the residual on the Dirichlet nodes to be equal to a multiple of the current solution.  With this
scheme, all rows and columns of the Jacobian corresponding to Dirichlet nodes are zero except for a
single diagonal entry, thus the system retains symmetry and satisfaction of the Dirichlet conditions
does not interfere with solving the other equations.  For good multigrid performance, the diagonal
entry should be similar to the diagonal entry of the Jacobian for nearby nodes.  To ensure this, we
set the residual at Dirichlet nodes to
\begin{align}\label{eq:dirichlet-scale}
  f_u &= 2 \eta (4 h_yh_z/h_x + h_x h_z/h_y + h_x h_y/h_z) u \\
  f_v &= 2 \eta (h_yh_z/h_x + 4 h_x h_z/h_y + h_x h_y/h_z) u
\end{align}
where $h_x,h_y,h_z$ are the local element dimensions.  This scaling produces the same diagonal
entries that would appear if the domain was extended so that constant viscosity momentum equations
appeared at the formerly Dirichlet nodes.

\subsection{Achieving high throughput}

\section{Numerical examples}\label{sec:examples}

\section{Discussion}

\nonumber\section{Acknowledgments} We thank the PETSc team for support and a framework in which this
work could be naturally expressed.  This work was supported by Swiss National Science Foundation
Grant 200021-113503/1.

\bibliographystyle{amsalpha}
\bibliography{flow}
\end{document}
