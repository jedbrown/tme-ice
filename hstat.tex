\documentclass[final]{siamltex}
\usepackage{amsmath,amssymb,bm}
\usepackage{graphicx,verbatim,multirow}
\usepackage{microtype}
\usepackage{ucs}
\usepackage[utf8x]{inputenc}
\usepackage{siunitx}
\usepackage{booktabs}
\usepackage{hyperref}
%\usepackage{natbib} % natbib itself has some problem with siam.bst
\usepackage{twoopt}
\usepackage{ifthen}
\providecommand{\citet}[2][]{{\ifthenelse{\equal{#1}{}}{\cite{#2}}{\cite[#1]{#2}}}}
\providecommandtwoopt{\citep}[3][][]{{\ifthenelse{\equal{#1}{}}{}{#1 }\ifthenelse{\equal{#2}{}}{\cite{#3}}{\cite[#2]{#3}}}}
\newcommand\email[1]{{\href{mailto:#1}{\nolinkurl{#1}}}}

% \usepackage[displaymath]{lineno}
% \linenumbers*[1]
% % The idiot that wrote this package didn't make it work with amsmath.
% % http://www.latex-community.org/forum/viewtopic.php?f=5&t=163#
% \newcommand*\patchAmsMathEnvironmentForLineno[1]{%
%   \expandafter\let\csname old#1\expandafter\endcsname\csname #1\endcsname
%   \expandafter\let\csname oldend#1\expandafter\endcsname\csname end#1\endcsname
%   \renewenvironment{#1}%
%      {\linenomath\csname old#1\endcsname}%
%      {\csname oldend#1\endcsname\endlinenomath}}% 
% \newcommand*\patchBothAmsMathEnvironmentsForLineno[1]{%
%   \patchAmsMathEnvironmentForLineno{#1}%
%   \patchAmsMathEnvironmentForLineno{#1*}}%
% \AtBeginDocument{%
% \patchBothAmsMathEnvironmentsForLineno{equation}%
% \patchBothAmsMathEnvironmentsForLineno{align}%
% \patchBothAmsMathEnvironmentsForLineno{flalign}%
% \patchBothAmsMathEnvironmentsForLineno{alignat}%
% \patchBothAmsMathEnvironmentsForLineno{gather}%
% \patchBothAmsMathEnvironmentsForLineno{multline}%
% }

\newcommand{\EE}{\mathcal E}
\newcommand{\KK}{\mathsf K}
\newcommand{\PP}{\mathsf P}
\newcommand{\VV}{\bm V} 
\newcommand{\R}{\mathbb R}
\newcommand{\ASM}{\mathrm{ASM}}
\newcommand{\RASM}{\mathrm{RASM}}
\newcommand{\bigO}{{\mathcal{O}}}
\newcommand{\abs}[1]{{\left\lvert #1 \right\rvert}}
\newcommand{\norm}[1]{{\left\lVert #1 \right\rVert}}
\newcommand{\tcolon}{{ : }}
\newcommand{\ip}[2]{{\left\langle #1, #2 \right\rangle}}

\usepackage{xspace}
\makeatletter
\DeclareRobustCommand\onedot{\futurelet\@let@token\@onedot}
\def\@onedot{\ifx\@let@token.\else.\null\fi\xspace}
\def\eg{{e.g}\onedot} \def\Eg{{E.g}\onedot}
\def\ie{{i.e}\onedot} \def\Ie{{I.e}\onedot}
\def\cf{{c.f}\onedot} \def\Cf{{C.f}\onedot}

\DeclareMathOperator{\sspan}{span}
\DeclareSIUnit\year{a}
\sisetup{retain-unity-mantissa = false}

% \authorrunninghead{BROWN ET AL.}
% \titlerunninghead{TME FOR HYDROSTATIC ICE SHEET FLOW}

\title{Achieving textbook multigrid efficiency for hydrostatic ice sheet flow}
\author{%
  {Jed Brown}\thanks{Laboratory of Hydraulics, Hydrology, and Glaciology (VAW), ETH Z\"urich, 8092 Z\"urich, Switzerland (\email{brown@vaw.baug.ethz.ch})}
  \and {Barry Smith}\thanks{Mathematics and Computer Science Division, Argonne National Laboratory, Argonne, IL 60439, USA (\email{bsmith@mcs.anl.gov})}
  \and {Aron Ahmadia}\thanks{Supercomputing Laboratory, King Abdullah University of Science and Technology, Thuwal, Makkah, Saudi Arabia (\email{aron.ahmadia@kaust.edu.sa})}}
\begin{document}
\maketitle
\begin{abstract}
The hydrostatic equations for ice sheet flow offer improved fidelity compared to the shallow ice approximation and shallow stream approximation (SSA) popular in today's ice sheet models. Nevertheless, they present a serious bottleneck because they require the solution of a 3D nonlinear system, as opposed to the 2D system present in SSA.  This 3D system is posed on high-aspect domains with strong anisotropy and variation in coefficients, making it expensive to solve using current methods.  This paper presents a Newton-Krylov multigrid solver for the hydrostatic equations that demonstrates textbook multigrid efficiency (an order of magnitude reduction in residual per iteration and solution of the fine-level system at a small multiple of the cost of a residual evaluation).  Scalability on Blue Gene/P is demonstrated, and the method is compared to various algebraic methods that are in use or have been proposed as viable approaches.
\end{abstract}
\begin{keywords}
  hydrostatic, ice sheet, Newton-Krylov, multigrid, preconditioning
\end{keywords}
\pagestyle{myheadings}
\thispagestyle{plain}
\markboth{BROWN, SMITH, AND AHMADIA}{TEXTBOOK MULTIGRID EFFICIENCY FOR HYDROSTATIC ICE SHEET FLOW}

\section{Introduction}
The dynamic response of ice streams and outlet glaciers is poorly represented using the shallowness
assumptions inherent in the present generation of ice sheet models.  Accurate simulation of this
response is crucial for prediction of sea level rise; indeed, the inability of available models, based on
the shallow ice approximation (SIA)~\citep{hutter1983tgm} and shallow stream approximation
(SSA)~\citep{morland1987unconfined,weis1999theory}, to simulate these processes was cited as a major
deficiency in the Fourth Assesment Report of the Intergovernmental Panel on Climate Change~\citep{ipcc2007ar4-syr}.

The hydrostatic equations were introduced by \citet{blatter1995vas} as a model of intermediate complexity between the full non-Newtonian Stokes system and the vertically-integrated SIA and SSA models.
A more precise analysis, including the limiting cases of fast and slow sliding, was given in \citet{schoof2010thin}.
Well-posedness was proven in \citet{colinge1999strongly}, and approximation properties of finite-element methods were analyzed in \citet{glowinski2003approximation,chow2004finite}.
\citet{hindmarsh2004numerical} conducted a perturbation analysis in one horizontal dimension to study validity of different continuum models including SIA, hydrostatic, and Stokes.
\citet{pattyn2008beh} contains a model intercomparison using several implementations of the hydrostatic equations for periodic benchmark problems with large amplitude basal topography and stickyness variability.
The hydrostatic equations were used for transient simulation in \citet{pattyn2002tgr} and in 3D models in \citet{pattyn2003ntd}, as well as subsequent work.
These equations have been known by a variety of names in the literature including ``Blatter'', ``Pattyn'', ``Blatter-Pattyn'', ``LMLa''~\citep{hindmarsh2004numerical}, and ``first order''~\citep{greve2009dynamics}.

The use of hydrostatic equations in current models has been limited, however, by the cost of solving the 3D nonlinear system for velocity.  This cost comes from both slow convergence on the
nonlinearities (rheology and slip) and expensive linear solves using standard preconditioners such
as incomplete factorization and one-level domain decomposition.  The poor linear solve performance is
attributable to the strong anisotropy and heterogeneity imposed by the rheology and geometry.

In the present work, we introduce a Newton-Krylov multigrid solver that demonstrates textbook
multigrid efficiency, characterized by convergence in a small multiple of the cost of a single,
fine-level residual evaluation, and typically involving an order of magnitude reduction in residual
per multigrid (V or F) cycle.  The scheme converges quadratically on the nonlinearities, is rapidly
globalized by using grid sequencing, is robust to parameters and geometry, coarsens rapidly in
almost all cases, and exhibits excellent parallel scalability.  Our code is freely available as part
of the Portable Extensible Toolkit for Scientific computing (PETSc)~\citep{petsc-web-page}.

Section~\ref{sec:equations} presents the equations and discretization, Section~\ref{sec:solver} describes the solver, and Section~\ref{sec:hstatexamples} demonstrates performance and scalability with numerical examples. Section 5 summarizes our conclusions.

\section{Equations and Discretization}\label{sec:equations}
The hydrostatic equations are obtained from the non-Newtonian Stokes equations in the limit where
horizontal derivatives of vertical velocity are small.  Neglecting these terms allows
incompressibility to be enforced implicitly by eliminating pressure and vertical velocity, leaving a
system involving only horizontal components of velocity.  See \citet{schoof2010thin} for a rigorous
derivation and asymptotic analysis.

Consider an ice domain $\Omega \subset \R^3$ lying between a Lipschitz continuous bed $b(x,y)$ and surface $s(x,y)$ with thickness $h = s-b$ bounded below by a positive constant.
\footnote{The singular limit $h\to 0$ is important in the case of grounded margins, but the present work does not pursue it.}
The
velocity $\bm u = (u,v) \in \bm V = \bm W^{1,1+1/n}(\Omega)$ satisfies conservation of momentum, which,
omitting inertial and convective terms as is standard for ice sheets, is given by
\begin{align}\label{eq:momentum}
  - \nabla\cdot \left[ \eta
  \begin{pmatrix}
    4 u_x + 2 v_y & u_y + v_x & u_z \\
    u_y + v_x & 2 u_x + 4 v_y & v_z
  \end{pmatrix} \right] + \rho g \nabla s & = 0,
\end{align}
where
\begin{align}\label{eq:viscosity}
  \eta(\gamma) = \frac B 2 \left(\epsilon^2/2 + \gamma\right)^{\frac{1-n}{2n}}
\end{align}
is the nonlinear effective viscosity with regularizing strain rate $\epsilon$ and
\begin{align*}
  \gamma = u_x^2 + v_y^2 + u_xv_y + \frac 1 4 (u_y+v_x)^2 + \frac 1 4 u_z^2 + \frac 1 4 v_z^2
\end{align*}
is the second invariant.  Ice sheet models typically take $n=3$ as the power law exponent.  Equation
\eqref{eq:momentum} is subject to natural boundary conditions at the free surface and either no-slip
$\bm u = 0$ or power-law slip with friction parameter
\begin{align*}
  \beta^2(\gamma_b) = \beta_0^2 \left(\epsilon_b^2/2 + \gamma_b\right)^{\frac{m-1}{2}},
\end{align*}
where $\gamma_b = \frac 1 2 (u^2 + v^2)$, $\epsilon_b$ is regularizing velocity, and $m \in (0,1]$ is the
exponent that produces Navier slip for $m=1$, Weertman~\citep{weertman1957sliding} sliding for
$m=1/n$, and Coulomb slip as $m \to 0$.  In the present work, we define $\epsilon$ and $\epsilon_b$ using a strain rate of $\SI{1e-5}{\per\year}$ and a slip velocity of $\SI{1}{\metre\per\year}$, respectively.

To discretize this system with a finite-element method, we introduce the test functions $\bm \phi
\in \VV$ and integrate by parts to produce the weak form: Find $\bm u \in \VV$ such that
\begin{align}\label{eq:weak}
  \int_\Omega \nabla\bm\phi \tcolon \eta \bm 1 \tcolon
  \begin{pmatrix}
    4 u_x + 2 v_y & u_y + v_x & u_z \\
    u_y + v_x & 2 u_x + 4 v_y & v_z
  \end{pmatrix} + \bm\phi \cdot\rho g \nabla s
  + \int_{\Gamma_{\text{bed}}} \bm \phi \cdot \beta^2(\abs{\bm u}^2/2) \bm u = 0
\end{align}
for all $\bm \phi \in \VV$, where $\Gamma_{\text{bed}}$ is the slip portion of the bed.

For our numerical studies, equation \eqref{eq:weak} was discretized on a topologically structured hexahedral grid using $Q_1$ finite elements and standard $2^3$-point Gauss quadrature.  Length, time, and mass units were chosen so that thickness, velocity, and driving stresses are of order 1.  See Section~\ref{ssec:dirichlet} for details on the enforcement of Dirichlet boundary conditions.

The source code for our implementation is distributed as an example in PETSc~\citep{petsc-web-page} versions 3.1 and later.
\footnote{
Upon unpacking the source, which can be downloaded from \url{http://mcs.anl.gov/petsc}, see \texttt{src/snes/examples/tutorials/ex48.c.}
}
Several generalizations of the tests from ISMIP-HOM~\citep{pattyn2008beh} are implemented, run with the option \texttt{-help} for a complete list of options.  Incidentally, our results for ``\SI{5}{\kilo\metre} test $C$'' in that paper (variable slipperiness on a flat bed) agree to three significant figures with the ``Stokes'' results therein.
This is consistent with the asympotic analysis of \citet{schoof2010thin}, which shows that slipperiness perturbations are not present in the leading-order error terms for the hydrostatic equations (which are purely geometric); \cf \citet[Table~4 and Figure~8]{pattyn2008beh}, where the ensemble range is nearly as large as the mean.

\section{Solver and Implementation}\label{sec:solver}
We begin by writing the discretization of \eqref{eq:weak} as an algebraic system $F(U) = 0$ with
Jacobian $J(U)$.  This nonlinear system is solved with a Newton iteration that requires an
approximate solution $\delta U$ of
\begin{equation}\label{eq:newton-step}
  J(U)\delta U = -F(U) .
\end{equation}
Newton methods are quadratically convergent in the terminal phase but may converge slowly or
not at all in early phases.  Many applications of the present solver are in a time-stepping code
where the initial iterates start within the region of quadratic convergence; thus globalization
would rarely be a concern. But since a good initial iterate is not available in the present tests,
we use grid sequencing (solving the problem on a sequence of coarser grids) to produce an initial
iterate on the fine grid.
Globalization is also a critical issue when solving steady-state problems.
Grid sequencing requires a geometric hierarchy of meshes with interpolation operators to move the solution to the next finer level.
Managing this hierarchy is often seen as a programming burden, but it exposes more robust algorithms than are available otherwise.

The Newton step \eqref{eq:newton-step} is solved by a Krylov method such as GMRES, for which the iteration count is highly dependent on the quality of the preconditioner.  Since $J(U)$ is symmetric positive definite (SPD), methods such as conjugate gradients could be used. This work always uses GMRES, however, because it allows the use of nonsymmetric preconditioners, and the iteration counts are always kept low so that storage and restarts are not an issue.  As an SPD system, it has a wide variety of preconditioners to choose from; however, viscosity contrasts and strong anisotropy cause most preconditioners to perform poorly.  The rest of this section describes the methods used to produce a scalable algorithm in spite of these difficulties.

\subsection{Anisotropic Meshes}
The ratio of width to thickness for outlet glaciers (the regions of ice sheets with greatest physical interest) ranges from order 1 to over $100$.  The nonlinear constitutive relation \eqref{eq:viscosity} produces three to four orders of magnitude variation in viscosity (usually with fastest variation in the vertical) and the Newton linearization of \eqref{eq:weak} produces additional anisotropy, effectively collapsing the conductivity tensor in the direction of the velocity gradient.

For systems with a priori known anisotropy, semi-coarsening has been successful for attaining
satisfactory multigrid performance even with weak smoothers like SOR, but semi-coarsening is
unattractive for two reasons.  First, semi-coarsening in the vertical direction would necessitate
many levels because it reduces the problem size by only a factor of 2 on each coarsening (instead of
8 for isotropic coarsening), and the fully coarsened problem would still be far too large for a
direct solver, necessitating further coarsening in the horizontal direction.  The presence of many
levels leads to more synchronization in parallel, which is detrimental to scalability and makes
performance more sensitive to network latency.  Second, viscosity contrasts and anisotropy
unaligned with the grid arise when the friction parameter $\beta^2$ is not smooth, as is the standard
case when studying the migration of ice stream margins as the bed transitions from frozen (no-slip
or very high friction) to temperate and very slippery depending on subglacial hydrology.

To coarsen the system isotropically even on high-aspect domains, we order the unknowns so that columns
are contiguous with a natural block size of 2
(\ie $\{u_{i,j,k},v_{i,j,k},u_{i,j,k+1},v_{i,j,k+1},\dotsc\}$ where $k$ is the index that is
increasing in the vertical direction) and not decomposed in parallel.  This decomposition is
reasonable since the number of vertical levels used in simulations is typically between 10 and 100;
it also is convenient since it is compatible with decompositions used by other climate model components.

With this ordering, zero-fill incomplete factorization effectively performs an exact solve of the
column since all the neglected fill (relative to an exact factorization) is caused by the coupling
with adjacent columns.  Pure line smoothers were also tried as a smoother on the finest level, but
robustness was significantly impacted, and the memory benefits were deemed insufficient to pursue
further.

In scenarios with little sliding and gentle geometry compared to the ice thickness, the ``shallow ice approximation'' (SIA) is valid, allowing the velocity field to be determined locally from the surface slope and a column integral.
If horizontal derivatives were discarded from the linearized hydrostatic equations, the corresponding matrix would have only uncoupled tridiagonal systems for each column and the solution of these systems would be equivalent to linearized SIA.
These tridiagonal systems become singular as basal friction $\beta^2 \to 0$, and furthermore, SIA is physically inconsistent if $\beta^2$ is discontinuous even when it is bounded below by a sufficiently large constant due to discontinuous velocities \citep{fowler2001modelling}.
% Fowler: ``There is a malaise afoot which consists of believing that if a numerical solution `looks' all right, it is. A discontinuous velocity is simply nonsense.''
The contribution to the hydrostatic matrix from discrete horizontal derivatives is small (order $h_z^2/h_x^2$) relative to that from vertical derivatives when the element aspect ratio ($h_z/h_x$) is small, and therefore high energy modes of the hydrostatic matrix, corresponding to velocity variation primarily in the vertical, become well approximated by the block tridiagonal system representing SIA.
When the bed is very slippery, there are medium to long wavelength modes with vanishing energy in the norm induced by the (nearly singular) SIA equations, but non-vanishing energy in the norm induced by the hydrostatic equations.
These modes are problematic if the tridiagonal SIA system is used to precondition hydrostatic because it will take many iterations for the Krylov solver to locate all of these modes not controlled by the preconditioner.

Incomplete factorization with column ordering provides nearly exact coupling in the vertical, effectively containing the tridiagonal SIA plus some coupling with nearby columns, and is an effective stand-alone preconditioner when the bed is sticky and horizontal grid spacing is large, despite lack of a coarse level.
Indeed, with a typical ice thickness of \SI{1}{\kilo\metre} resting on a frozen bed and elements \SI{5}{\kilo\metre} on
a side, block Jacobi with zero-fill incomplete Cholesky converges to relative tolerance of $10^{-5}$
in about 10 Krylov iterations independent of the horizontal extent of the domain (number of elements
in the horizontal), independent of the number of elements in the vertical, and independent of the
number of subdomains (provided they do not get too small: there is some degredation when subdomain
size approaches a single column).  However, none of these favorable performance characteristics
remain when the elements become small relative to the ice thickness or when the bed becomes
slippery since the usual $\bigO\big((L/H)^2\big)$ condition number for second-order elliptic problems
preconditioned by one-level Schwarz methods with subdomains of size $H$, see \citet{smith1996domain}, becomes apparent.
Indeed, we have found low-fill incomplete factorization to be nearly unusable as part of a one-level additive Schwarz method for problems with slippery beds or steep geometry, even at low resolution, as investigated in Section~\ref{sec:1level}.

\subsection{Dirichlet Boundary Conditions}\label{ssec:dirichlet}
Multigrid is often sensitive to the enforcement of boundary conditions.  Ideally, Dirichlet
conditions would be completely removed from the solution space, but doing so complicates grid
management on structured grids, so instead we leave these degrees of freedom in the system but
decouple them from the other equations.  During residual evaluation in the finite-element context,
this strategy corresponds to evaluating integrals with the Dirichlet condition satisfied exactly and setting
the residual on the Dirichlet nodes to be equal to a multiple of the current solution.  With this
scheme, all rows and columns of the Jacobian corresponding to Dirichlet nodes are zero except for a
single diagonal entry. Thus the system retains symmetry, and satisfaction of the Dirichlet conditions
does not interfere with solving the other equations.  For good multigrid performance, the diagonal
entry should be similar to the diagonal entry of the Jacobian for nearby nodes.  To ensure this, we
set the residual at Dirichlet nodes to
\begin{equation}\label{eq:dirichlet-scale}
  \begin{split}
    f_u &= 2 \eta h_xh_yh_z (4/h_x^2 + 1/h_y^2 + 1/h_z^2) u \\
    f_v &= 2 \eta h_xh_yh_z (1/h_x^2 + 4/h_y^2 + 1/h_z^2) v,
  \end{split}
\end{equation}
where $h_x,h_y,h_z$ are the local element dimensions.  This scaling produces the same diagonal
entries that would appear if the domain was extended so that constant viscosity momentum equations
appeared at the formerly Dirichlet nodes.

\subsection{Matrices}
The most expensive operations are Jacobian assembly and sparse matrix kernels such as matrix-vector multiplication and solves with incomplete triangular factors.
The former involves evaluation of fractional powers and finite element quadrature loops.  While fractional powers take most of the time for residual evaluation, they are less significant than quadrature loops for assembly.  The quadrature loops were explicitly vectorized by using SSE2 intrinsics which led to a 30\% speedup on Core 2 and Opteron architectures using both GNU and Intel compilers.  There was no manual vectorization for the Blue Gene/P results quoted in Section~\ref{sec:pscaling}.

Assembly costs could be further mitigated by recomputing the matrix less frequently, either by using a modified Newton method (degrades nonlinear convergence rate) or by applying the current operator matrix-free by finite differencing the residual or using automatic differentiation in which case only the preconditioner is lagged.
These are runtime options in the present code; but since they do not offer a clear benefit they have not been pursued in the present work.
If matrix-free application of the true Jacobian is used, several other preconditioning options become available without impacting the nonlinear convergence rate.
One could assemble only the block-tridiagonal column coupling, ignoring horizontal coupling, thus saving the memory for the finest level(s).
Additionally, a truly 2D coarse problem can be defined by using the shallow stream equations~\citep{morland1987unconfined,weis1999theory,schoof2006variational} and restriction operators defined by integrating the entire column.
These possibilities are also runtime options, but have not exhibited a level of robustness comparable to that of the more conventional methods pursued here.

The Jacobian is always symmetric positive definite and has a natural block size of 2, so we use a symmetric block format (PETSc's SBAIJ).  This format stores one column index per $2\times 2$ block in the upper triangular part of the matrix and therefore uses about half the storage of the nonsymmetric BAIJ format, which in turn uses 25\% less memory than a scalar format (AIJ). Multiplication for symmetric storage requires twice as much parallel communication as nonsymmetric storage, albeit with the same number of messages. In return, the diagonal part of a parallel decomposition does twice as much work per matrix entry and thus achieves higher throughput, as shown in Table~\ref{tab:sbaij}.

\begin{table}
  \centering\caption{Throughput (Mflop/s) for different matrix formats on Core 2 Duo (P8700) and Opteron 2356 (two sockets). \texttt{MatSolve} is a forward- and back-solve with incomplete Cholesky factors.  The AIJ format is using ``inodes'' which unrolls across consecutive rows with identical nonzero pattern (pairs in this case).}\label{tab:sbaij}
  \begin{tabular}{l rrr rrr}
    \toprule
    \multirow{3}{*}{Kernel} & \multicolumn{3}{c}{Core 2, 1 process} & \multicolumn{3}{c}{Opteron, 4 processes} \\
    \cmidrule(r){2-4} \cmidrule(l){5-7}
                      & AIJ & BAIJ & SBAIJ & AIJ  & BAIJ & SBAIJ \\
    \midrule
    \texttt{MatMult}  & 812 & 985  & 1507  & 2226 & 2918 & 3119  \\
    \texttt{MatSolve} & 718 & 957  & 955   & 1573 & 2869 & 2858  \\
    \bottomrule
  \end{tabular}
\end{table}

There are two ways to construct matrices on the coarse levels of multigrid methods.
The first, which we use in almost all our numerical examples, is to rediscretize the system on the coarse mesh.
In our implementation, this involves re-evaluating nonlinearities on each level of the hierarchy, although restricting fine-level coefficients of the linearized problem would also be possible.
This procedure produces coarse operators that are as sparse as possible on each of the levels.
The Galerkin procedure is an alternative that is mandatory for algebraic multigrid.
Given an interpolation operator $P: \VV_{\text{coarse}} \to \VV_{\text{fine}}$ and assembled fine-level matrix $A_{\text{fine}}$, the Galerkin coarse operator is $A_{\text{coarse}} = P^T A_{\text{fine}} P$.
These operators work well for some problems, but computing the sparse matrix product in parallel involves significant communication and irregular memory access, so it is relatively expensive.
Additionally, second-order-accurate interpolation operators cause a loss of sparsity in the coarse-level operators, an effect known as stencil growth.
Stencil growth tends to blur regions where the solution has local structure and usually reduces the effectiveness of inexpensive smoothers.

\section{Numerical Examples}\label{sec:hstatexamples}

We present several numerical examples that demonstrate the algorithmic and parallel scalability of the Newton-Krylov multigrid approach.

\subsection{Algorithmic Scalability}\label{sec:ascaling}
We consider three model problems inspired by the periodic domain ISMIP-HOM~\citep{pattyn2008beh} tests.  All use surface $s(x,y) = -x\sin \alpha$, where $\alpha$ is the surface slope (the coordinate system is not rotated) and a bed similar to $b_A(x,y) = s(x,y) - \SI{1000}{\metre} + \SI{500}{\metre} \cdot \sin\hat{x}\sin\hat{y}$ for $(x,y) \in [0,L)^2$ with $\hat{x} = 2\pi x/L,\hat{y}=2\pi y/L$.  Test $X$ uses bed $b_X = b_A$ and stickiness parameter
\begin{align*}
  \beta_X^2(x,y) =
  \begin{cases}
    \SI{2000}{\pascal\year\per\metre}, & \text{if } r = \abs{(\hat x,\hat y) - (\pi,\pi)} < 1 \\
    0, & \text{otherwise}
  \end{cases}
\end{align*}
which is free slip except for a sticky circle at the center of the domain, which is not aligned with the grid.
A solution cutout for $L = \SI{10}{\kilo\metre}$ is shown in Figure~\ref{fig:xshear}.
This problem exhibits shear localization at the edges of the sticky region and is most extreme at high aspect ratio.
We choose $L = \SI{80}{\kilo\metre}$ and $\alpha = 0.05^\circ$ which produce velocities from \SIrange{0.9}{47}{\kilo\metre\per\year}.
\footnote{
This problem may be run with the options \texttt{-thi\_hom X -thi\_L 80e3 -thi\_alpha 0.05}, the other cases can be selected with similar options.
}
To demonstrate algorithmic scalability as the problem size is increased, this test was run on 8 processors starting from a coarse grid of $16\times 16\times 1$, refining twice in the horizontal by factors of 2 in both $x$ and $y$, then three times in the vertical by factors of 8 each to reach a fine mesh of $64\times 64\times 513$, which has elements of nominal dimension $1250\times 1250\times 1.95$ meters.

\begin{figure}
  \centering\includegraphics[width=0.8\textwidth]{x-shear}
  \caption{A cutout colored by velocity magnitude for flow over the bumpy bed of test $X$ at $L=\SI{10}{\kilo\metre}$ with $m=1/10$ nearly plastic basal yield model.
    The cut on the left shows along-flow velocity as the ice hits the sticky region, the cut on the right shows across-flow shear structure.}\label{fig:xshear}
\end{figure}

A visual representation of the nonlinear solve process is shown in Figure~\ref{fig:xgridseq}.
In this example and the next one, Luis Chac\'on's variant of the Eisenstat-Walker~\citep{eisenstat1996cft} method was used to automatically adjust linear solve tolerances as the nonlinear solve converges.  Solving the linear systems to higher tolerance would have little impact on the number of nonlinear iterations and would be visible in the form of more $\times$ marks below the solid lines.  That most $\times$ marks lie on the solid line for nonlinear residual is an indication that effort is well balanced between linear and nonlinear solves.
Note that approximately 10 linear V-cycles on the fine level are required to reduce the residual by 10 orders of magnitude.
We remark that Picard iteration takes at least 50 iterations to reach this tolerance (cf\onedot \citet{desmedt2010using} in which hundreds or thousands of iterations were needed for an easier problem).
When used in a time-stepping code, the solution at the last time step is available so typically only the terminal convergence rate is important which further favors Newton methods.
For example, an initial guess might have a residual $10^3$ higher than the required tolerance which takes 15 Picard iterations to solve, but only one Newton iteration.
Additionally, each linear solve for this fine-level problem requires hundreds or thousands of iterations with a one-level additive Schwarz method; see Section~\ref{sec:1level}, which considers a smaller problem.

\begin{figure}
  \centering\includegraphics{x-80km-m16p2l6-ew}
  % petscplot -m paper -t snes --solve-spec '1:' x.80km.m16p2l6.ew.log -o ../figures/x-80km-m16p2l6-ew.pdf
  \caption{Grid-sequenced Newton-Krylov solution of test $X$.  The solid lines denote nonlinear iterations, and the dotted lines with $\times$ denote linear residuals.}\label{fig:xgridseq}
\end{figure}

Test $Y$ places a \SI{200}{\metre} tower with vertical walls on the top of each bedrock bump and uses an uncorrelated but smoothly varying stickiness resembling a dimpled sombrero:
\begin{align*}
  b_Y(x,y) &=
  \begin{cases}
    b_A(x,y), & \text{if } b_A(x,y) < \SI{-700}{\metre} \\
    b_A(x,y) + \SI{200}{\metre}, & \text{otherwise}
  \end{cases} \\
  \beta_Y^2(x,y) &= \SI{1000}{\pascal\year\per\metre} \cdot (1 + \sin(\sqrt{16r})/\sqrt{10^{-2} + 16r}\ \cos\frac{3\hat x}{2}\, \cos\frac{3\hat y}{2}.
\end{align*}
This tests the quality of the coarse grids even when large geometric errors are committed.  Note that the hydrostatic equations cannot be considered valid in this regime since the topography is too abrupt.  Such topography is present in reality, however, so we may still desire an efficient solver.  Figure~\ref{fig:testy} depicts the solve for this problem in a \SI{10}{\kilo\metre} square domain. Because of the successively better resolution of the ``cliff,'' performance deteriorates on each level, as can be seen by the closer spacing of linear solve marks $(\times)$.  It is entirely acceptable up to level 3, however, where the elements are approximately \SI{12}{\metre} thick and stretch to reach over a \SI{200}{\metre} vertical cliff in \SI{125}{\metre} horizontal, a slope of \SI{58}{\degree}.  On the finest level, they are \SI{6}{\metre} thick and stretch over the cliff in \SI{62}{\metre} horizontal, a slope of \SI{73}{\degree}.  The approximation properties of such elements are poor and, considering that the continuum equations are invalid here, we believe this resolved topography is significantly rougher than will needed in applications.

\begin{figure}
  \centering\includegraphics{y-10km-m10p6l5-ew}
  % petscplot -m paper -t snes --solve-spec '1:' y.10km.m10p6l5.ew.log -o ../figures/y-10km-m10p6l5-ew.pdf
  \caption{Grid sequenced Newton-Krylov convergence for test $Y$.}\label{fig:testy}
\end{figure}

Test $Z$ sets $b_Z = b_A$, $\beta_Z^2 = \beta_X^2$, and nonlinear sliding with exponent $m=0.3$. It is a regime where the hydrostatic equations are valid, provided the wavelength $L$ is not too small.  We use this case to explore linear solve performance in Figure~\ref{fig:linear}.

\begin{figure}
  \centering\includegraphics{linear4}
  % ~/petscplot/petscplot -m paper --width-pt 380 -t algorithmic z.m6p4l5.newton.icc.log : z.m6p4l5.picard.asm8.icc1.log : z.m6p4l6.picard.icc.log : z.m6p4l6.mult.n8.o0.r2.log --legend-labels 'Newton, ICC(0), serial:Picard ASM(1)/ICC(1), 8 subdomains:Picard ICC(0) serial:Newton, V-cycle, 8 subdomains' -o ../figures/linear4.pdf
  \caption{Average number of Krylov iterations per nonlinear iteration.  Each nonlinear system was solved to a relative tolerance of $10^{-2}$.}\label{fig:linear}
\end{figure}

\subsection{Parallel Scalability on Blue Gene/P}\label{sec:pscaling}
Two types of parallel scalability are considered.
Strong scalability is the ability to solve a fixed problem size faster by using more processors.
It is represented by a log-log plot of time versus number of processors where a slope of $-1$ is optimal.
Weak scalability is the ability to solve larger problems in constant time by using more processors with the problem size per processor remaining fixed.
It is represented by a plot of time versus number of processors where a slope of $0$ is optimal.
Strong scalability is important in transient simulations with many time steps while weak scalability is more important for attaining high resolution in steady-state simulations or transient simulations with fewer time steps.
Optimal strong and weak scalability together would imply that a problem with $N$ elements could be solved in $\bigO(N/P)$ time on $P$ processors.

We investigate strong scaling using test $Z$ at \SI{80}{\kilo\metre} with basal friction exponent $m= 0.3$ on Shaheen, a Blue Gene/P at the KAUST Supercomputing Laboratory.
Figure~\ref{fig:shaheen-strong} shows strong scalability for four fixed global problem sizes with coarse meshes of $16\times 16\times 3$, $32\times 32\times 3$, $64\times 64\times 3$, and $64\times 64\times 1$.
The first three use five levels of isotropic refinement to reach target meshes of $256\times 256\times 48$, $512\times 512\times 48$, and $1024\times 1024\times 48$, the latter with nominal element sizes of $78\times 78\times 21$ meters.
The last coarse mesh is refined anisotropically and more aggressively to reach $1024\times 1024\times 64$ with only four levels.
The smallest two coarse problems are solved redundantly and the second is also solved using the ``$XX^T$'' direct solver of~\citet{tufo2001fast} (TFS) which exhibits significantly better scalability but the coarse-level problem still presents a bottleneck.
The largest two coarse problems are only solved approximately using a V-cycle of the algebraic multigrid package BoomerAMG~\citep{henson2002bpa}, which removes the severe scalability bottleneck of a direct coarse-level solve.
These coarse problems have been designed so that BoomerAMG is effective, see Section~\ref{sec:1level} for BoomerAMG applied to fine-level problems.

\begin{figure}
  \centering\includegraphics{shaheen-strong}
  \caption{Strong scaling on Shaheen for different size coarse levels problems and different coarse level solvers (see text for details).
    The straight lines on the strong scaling plot have slope $-1$ which is optimal.
    Grid sequencing is used, but only the nonlinear solve on the finest level is shown since strong scalability is most important when many time steps are needed.}\label{fig:shaheen-strong}
\end{figure}

Figure~\ref{fig:shaheen-weak} shows weak scalability.
The size of the coarse grid was held constant, and additional levels were added as the number of processes was increased, such that the subdomain sizes remain approximately constant.
As the resolution increases, the nonlinearity strength and linear stiffness also changes, explaining the relatively expensive solve with four processes.
This effect is also apparent in the Level 3 convergence of Figure~\ref{fig:xgridseq}.
Four phases of the solution process dominate the runtime and are identified in Figure~\ref{fig:shaheen-weak}.
Multigrid setup costs account for less time than nonlinear residual evaluation for the larger problem sizes.

\begin{figure}
  \centering\includegraphics{shaheen-weak}
  % Note: weak scaling for nonlinear problems is tricky unless the nonlinearity and linear stiffness are equally strong at all resolutions.  I don't know a good way to handle it.
  \caption{Weak scaling on Shaheen with a breakdown of time spent in different phases of the solution process.
    Times are for the full grid-sequenced problem instead of just the finest level solve.}\label{fig:shaheen-weak}
\end{figure}

\subsection{Algebraic Methods}\label{sec:1level}
Building a geometric hierarchy with rediscretization on coarse levels adds software complexity that many developers of numerical models do not want to deal with.  In this section, we summarize the performance characteristics of several popular algebraic methods.  We consider test $X$, $L = \SI{80}{\kilo\metre}, \alpha=0.03$, with $40\times 40\times 12$ elements distributed over four processes. We compare our multigrid method with several one-level domain decomposition methods, two algebraic multigrids, and field-split approaches. This problem is challenging for the standard Newton iteration which requires 37 iterations and should be accompanied by grid sequencing for efficiency. It takes the problem through a range of nonlinearities, however. Thus the number of Krylov iterations to solve with a relative tolerance of $10^{-5}$, presented below, is a good test of the linear solver.

We first consider one-level domain decomposition methods with incomplete factorization, which are currently used to solve the hydrostatic equations by \citep{seacism,issm} among others.  To keep iteration counts representative, we use full GMRES (no restart) with modified Gram-Schmidt orthogonalization (note that neither is practical for production use).  Conventional symmetric additive Schwarz is denoted $\ASM(k)$, where $k$ is the overlap, restricted additive Schwarz~\citep{cai1999restricted} is denoted $\RASM(k)$.  The average number of GMRES iterations per Newton iteration is shown in Table~\ref{tab:1level}.  Note that increasing overlap has no benefit when incomplete subdomain solvers are used.

\begin{table}
  \centering\caption{Average number of GMRES iterations per Newton iteration for one-level domain decomposition with different overlap and fill.}\label{tab:1level}
  \begin{tabular}{lcccc}
    \toprule
    \multirow{3}{*}{Decomposition} & \multicolumn{4}{c}{Subdomain solver} \\
    \cmidrule{2-5}
     & ICC(0) & ICC(1) & ICC(4) & Cholesky \\
    \midrule
    Block Jacobi & 367 & 315 & 220 & 97 \\
    $\ASM(1)$     & 508 & 441 & 296 & 59 \\
    $\RASM(1)$    & 368 & 306 & 190 & 52 \\
    $\ASM(2)$     & 521 & 445 & 316 & 44 \\
    $\RASM(2)$    & 365 & 305 & 189 & 38 \\
    \bottomrule
  \end{tabular}
\end{table}

The parallel algebraic multigrid packages ML~\citep{ml-guide} and BoomerAMG~\citep{henson2002bpa} provide potentially scalable alternatives.  ML is based on smoothed aggregation, tends to coarsen very rapidly, and provides its restriction and coarse-level matrices to PETSc so that that elaborate smoothers can be used.  ML does not converge for this problem with standard options; but with FGMRES on the outside of the V-cycle and GMRES(1) with $\RASM(1)$ as the smoother, using ICC(0) for the subdomain solve except on level 1, where a direct solve was used, we see 34 V-cycles per Newton iteration.  ML needs only three levels to reach a coarse level with 144 degrees of freedom.  BoomerAMG is a classical algebraic multigrid, which tends to coarsen slowly on anisotropic problems and does not expose the internal details, so smoother choices are limited.  BoomerAMG needs seven levels to reach a coarse grid with 663 degrees of freedom and averages 76 iterations per Newton iteration.  There were other, still somewhat challenging, problems for which BoomerAMG was competitive with geometric multigrid in terms of iteration count, but the setup costs and required number of levels was always large.  ML never exhibited low iteration counts, even for easy problems.

Another approach to solving multicomponent problems is to split the components and solve scalar problems for each in hopes that the scalar problems can be more readily handled by available software such as algebraic multigrid.  The split problems can be combined additively, multiplicatively, or symmetric multiplicatively.  Unlike most Schwarz methods, additive methods are not typically implemented to expose concurrency, but it is simpler to implement in a matrix-light way because only the ``self-coupling'' terms need to be made available.  Multiplicative methods need to apply the off-diagonal submatrix, and the most efficient way to do so is usually by assembling it; but the submatrix can also be applied by finite differencing the full residual.  The results, shown in Table~\ref{tab:fieldsplit}, are uninspiring, especially when considering that field-split creates additional synchronization points and attains lower throughput since it works with scalar matrices instead of block matrices (see Table~\ref{tab:sbaij}).

\begin{table}
  \centering\caption{Average number of GMRES iterations per Newton iteration for field-split preconditioners with different ways of combining the splits and different solvers within the splits.
BoomerAMG used 7 levels and ML had 3 with the same solver parameters as discussed in the text for the coupled approach.}\label{tab:fieldsplit}
  \begin{tabular}{lccc}
    \toprule
    \multirow{3}{*}{Solver in Splits} & \multicolumn{3}{c}{Method of combining splits} \\
    \cmidrule{2-4}
     & Additive & Multiplicative & Sym. Multiplicative \\
    \midrule
    Cholesky            & 19       & 9.9            & 9.3                 \\
    ML                  & 41       & 34             & 30                  \\
    BoomerAMG           & 89       & 83             & 78                  \\
    $\RASM(1)$+Cholesky & 186      & 173            & 84                  \\
    \bottomrule
  \end{tabular}
  %  for ft in additive multiplicative symmetric_multiplicative; do ~/usr/bin/mpiexec -n 4 ./e -M 40 -P 13 -thi_L 80e3 -thi_hom x -thi_alpha 0.03 -thi_nlevels 1 -{snes,ksp}_converged_reason -pc_type fieldsplit -pc_fieldsplit_type $ft -fieldsplit_0_pc_type ml -fieldsplit_0_pc_ml_maxnlevels 3 -fieldsplit_0_mg_levels_ksp_type gmres -fieldsplit_0_mg_levels_pc_type asm -fieldsplit_0_mg_levels_pc_asm_type restrict -fieldsplit_0_mg_levels_1_sub_pc_type cholesky -fieldsplit_1_pc_type ml -fieldsplit_1_pc_ml_maxnlevels 3 -fieldsplit_1_mg_levels_ksp_type gmres -fieldsplit_1_mg_levels_pc_type asm -fieldsplit_1_mg_levels_pc_asm_type restrict -fieldsplit_1_mg_levels_1_sub_pc_type cholesky; done
\end{table}

A good geometric multigrid for this problem uses four-levels with a coarse grid of $10\times 10\times 2$ elements, which is semi-refined twice by a factor of 2 in the horizontal, then by a factor of 6 in the vertical to reach the target $40\times 40\times 12$ grid.
The smoothers consist of a domain decomposition method and a subdomain solver that may be exact or inexact.
Direct solves for the subdomain problems on level 1 are inexpensive and tend to improve robustness so we always use a direct solve.
A different refinement involves a $10\times 10\times 1$ coarse grid and semi-refines twice by a factor of 2 in the horizontal, then by a factor of 12 in the vertical to reach the same target grid.
The coarse levels are smaller in this case so the refinement is more efficient provided the iteratation counts are similar.
Table~\ref{tab:xmg} explores a variety of multigrid preconditioner configurations with each coarse level.
A distinct effect is that inexact subdomain solvers cause minimal performance degradation; cf. Section \ref{tab:1level} where a factor of 5 to 10 degradation is visible.
Use of Galerkin coarse operators has a catastrophic effect on the iteration count and may help explain the poor robustness exhibited by the algebraic multigrids.
We cannot explain why symmetric additive Schwarz performs so poorly on this problem, but the other numbers are robust to changes in resolution, spatial domain, and number of processes.

\begin{table}
  \centering\caption{Average number of GMRES iterations per Newton for different multigrid preconditioners.}\label{tab:xmg}
  \begin{tabular}{l ll ll r@{.}l}
    \toprule
    \multirow{3}{*}{Coarse Problem} & \multicolumn{2}{c}{Level 1} & \multicolumn{2}{c}{Level 2 and 3} & \multicolumn{2}{c}{\multirow{3}{*}{Its}} \\
    \cmidrule(r){2-3} \cmidrule(l){4-5}
                                   & Decomp. & Subdomain & Decomp. & Subdomain         \\
    \midrule
    $10\times 10\times 2$ Redisc   & BJacobi & Cholesky & BJacobi & Cholesky & 8  & 9 \\
    $10\times 10\times 2$ Redisc   & BJacobi & Cholesky & BJacobi & ICC(0)   & 9  & 6 \\
    $10\times 10\times 2$ Redisc   & BJacobi & Cholesky & ASM(1)  & ICC(0)   & 11 & 9 \\
    $10\times 10\times 2$ Redisc   & BJacobi & Cholesky & RASM(1) & ICC(0)   & 6  & 9 \\
    $10\times 10\times 2$ Redisc   & ASM(1)  & Cholesky & ASM(1)  & ICC(0)   & 20 & 2 \\
    $10\times 10\times 2$ Redisc   & RASM(1) & Cholesky & RASM(1) & ICC(0)   & 5  & 9 \\
    $10\times 10\times 2$ Galerkin & RASM(1) & Cholesky & RASM(1) & ICC(0)   & 54 &   \\
    $10\times 10\times 1$ Redisc   & BJacobi & Cholesky & BJacobi & ICC(0)   & 10 & 3 \\
    $10\times 10\times 1$ Redisc   & ASM(1)  & Cholesky & BJacobi & ICC(0)   & 10 & 1 \\
    $10\times 10\times 1$ Redisc   & RASM(1) & Cholesky & BJacobi & ICC(0)   & 6  & 9 \\
    \bottomrule
  \end{tabular}
\end{table}

We remark that when using grid sequencing and the method in the last row of Table~\ref{tab:xmg}, the problem can be solved in seven Newton iterations on the fine level and an average of 5.4 V-cycles per Newton iteration.
If the Eisenstat-Walker method is used to avoid oversolving, the problem takes eight Newton iterations with a total of 12 V-cycles.

\section{Conclusion}
Several projects \citep{lemieux2011jfnk,issm,johnson2007modeling,desmedt2010using,pattyn2003ntd,lemieux2011jfnk} have developed software to simulate ice sheets using the hydrostatic equations.
All of these projects solve the resulting algebraic equations using one-level domain decomposition (if parallel) and incomplete factorization, and all but \citep{lemieux2011jfnk} use Picard linearization.
These methods are not scalable in the strong or weak sense so resolution and simulation turn-around time has been severely limited by the cost of the solver.

We have presented a grid-sequenced Newton-Krylov multigrid algorithm for solving the algebraic equations resulting from a finite element discretization of the hydrostatic equations.
This geometric multigrid method demonstrates textbook multigrid efficiency for extreme topography and basal conditions and offers thousandfold speedups relative to the methods currently in use.
Algebraic multigrid and field-split preconditioners were not found to be competitive in terms of robustness or efficiency.

The present velocity solver can be immediately incorporated in a semi-implicit ice sheet evolution model, but the resulting scheme is only conditionally stable (time steps must satisfy a CFL condition).
Fully implicit time integration offers an alternative that has no such constraint and is thus suitable for simulations over full ice ages and longer, but requires the solution of more difficult algebraic equations.
A prototype using the present velocity discretization has been developed, but scalable linear solvers for velocity-surface coupled problems is the subject of future investigation.

Several models involving only a vertically-integrated elliptic solve have been proposed as less costly alternatives to solving the hydrostatic equations, \eg \citep{bueler2009shallow,goldberg2009grounding,goldberg2010variational}.
A scalable solver has not yet been proposed for these vertically-integrated elliptic problems so the present model is faster than existing implementations at sufficiently high resolution.
To make a useful comparison, we estimate relative costs with the assumption that a similarly optimal solver exists for the vertically-integrated system, in which case the cost difference should be a constant factor independent of problem size and number of processors.
The ``hybrid'' model of \citet{goldberg2010variational} requires evaluating effective viscosity in 3D and integrating to produce depth-averaged viscosity which appears in the vertically-integrated equations.
Evaluating effective viscosity accounts for most of the cost of residual evaluation for the hydrostatic equations.
Therefore the cost of such an evaluation is comparable to the cost of the nonlinear viscosity evaluation required in a Picard iteration for the hybrid model of \citet{goldberg2010variational}.
Our solver produces high-accuracy solutions in about 30 times the cost of a residual evaluation, less if a good initial guess was available as when time stepping.
\citet{goldberg2010variational} uses a Picard iteration to solve the nonlinear system, requiring 50 iterations per nonlinear solve, therefore the hybrid solver is only likely to have similar cost to our hydrostatic solver when the linear solves require negligible time.
If the hybrid equations were instead solved by Newton iteration and the linear solves required similar or less time than evaluating viscosity, then the hybrid equations might be solved as much as 5 times faster than the hydrostatic equations.
\citet{bueler2009shallow} factors depth dependent velocity and temperature contributions out of the vertically-integrated nonlinear problem.
The process only applies if the constitutive relation can be factored and the resulting continuum equations have a lower formal order of accuracy with respect to shallowness and stickyness distribution than the model of \citet{goldberg2010variational}.
In exchange, the nonlinear elliptic solve can be completed without revisiting the 3D velocity field so the model cost has weaker dependence on the number of nonlinear iterations and thus potential to be a few times faster than the model of \citet{goldberg2010variational}.
Vertically-integrated models do not involve linear algebra operating on the 3D velocity field so they require significantly less memory than the hydrostatic equations when all operators are assembled.

The hydrostatic equations are well-suited to investigation of ice streams and outlet glaciers with gentle slope and arbitrary stickyness distribution.
For example, 3D ice stream thermodynamics (see \citet{raymond2000energy} for analytic and \citet{truffer2003isbrae} for numerical investigation with along-flow symmetry) could be reliably investigated using velocity defined by the hydrostatic equations.
The methods in this paper can efficiently solve the discrete hydrostatic equations in regimes with extreme basal topography and stickyness distributions.
The solution method is efficient even in regimes which are far in excess of where the continuum hydrostatic approximation is valid as an approximation to the Stokes equations.
The latter issue of continuum validity has been partially addressed using asymptotic analysis~\citep{schoof2010thin} and numerical comparison~\citep{hindmarsh2004numerical,pattyn2008beh}.
Despite being more complete than models involving only a vertically-integrated elliptic solve, the hydrostatic equations are not suitable for steep outlet glaciers like Jakoshavn Isbr{\ae} or Pine Island Glacier where solution of Stokes problems appears to be unavoidable.

\medskip
\begin{quote}\it
  Comprehensive log files and the scripts used to produce the figures in this paper are publicly available.
  They are currently hosted along with the source for this document at \url{https://github.com/jedbrown/tme-ice}.
\end{quote}


\section*{Acknowledgments}
We are grateful to Edward L. Bueler for his helpful commentary on an early draft of this paper.
This work was supported by Swiss National Science Foundation Grant 200021-113503/1, U.S. Department of Energy's Office of Science Ice Sheet Initiative for CL-imate ExtremeS program under Contract DE-AC02-06CH11357, and the Shaheen Supercomputing Laboratory at KAUST.

\bibliographystyle{siam}
\bibliography{jedbib/jedbib,petscapp}

\bigskip
% The text below can be removed in the published document
\begin{quotation}
The submitted manuscript has been created by UChicago Argonne, LLC,
Operator of Argonne National Laboratory (``Argonne'').  Argonne, a
U.S. Department of Energy Office of Science laboratory, is operated
under Contract No. DE-AC02-06CH11357.  The U.S. Government retains for
itself, and others acting on its behalf, a paid-up nonexclusive,
irrevocable worldwide license in said article to reproduce, prepare
derivative works, distribute copies to the public, and perform
publicly and display publicly, by or on behalf of the Government.
\end{quotation}
\end{document}
