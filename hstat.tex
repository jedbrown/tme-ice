\documentclass[final]{siamltex}
\usepackage{amsmath,amssymb,bm}
\usepackage{graphicx,verbatim,slashbox,multirow}
\usepackage{microtype}
\usepackage{ucs}
\usepackage[utf8x]{inputenc}
\usepackage{siunitx}
\usepackage{booktabs}

\usepackage{hyperref}
\newcommand{\citep}[1]{{\cite{#1}}}
\newcommand\email[1]{{\href{mailto:#1}{\nolinkurl{#1}}}}

% \usepackage[displaymath]{lineno}
% \linenumbers*[1]
% % The idiot that wrote this package didn't make it work with amsmath.
% % http://www.latex-community.org/forum/viewtopic.php?f=5&t=163#
% \newcommand*\patchAmsMathEnvironmentForLineno[1]{%
%   \expandafter\let\csname old#1\expandafter\endcsname\csname #1\endcsname
%   \expandafter\let\csname oldend#1\expandafter\endcsname\csname end#1\endcsname
%   \renewenvironment{#1}%
%      {\linenomath\csname old#1\endcsname}%
%      {\csname oldend#1\endcsname\endlinenomath}}% 
% \newcommand*\patchBothAmsMathEnvironmentsForLineno[1]{%
%   \patchAmsMathEnvironmentForLineno{#1}%
%   \patchAmsMathEnvironmentForLineno{#1*}}%
% \AtBeginDocument{%
% \patchBothAmsMathEnvironmentsForLineno{equation}%
% \patchBothAmsMathEnvironmentsForLineno{align}%
% \patchBothAmsMathEnvironmentsForLineno{flalign}%
% \patchBothAmsMathEnvironmentsForLineno{alignat}%
% \patchBothAmsMathEnvironmentsForLineno{gather}%
% \patchBothAmsMathEnvironmentsForLineno{multline}%
% }

\newcommand{\EE}{\mathcal E}
\newcommand{\KK}{\mathsf K}
\newcommand{\PP}{\mathsf P}
\newcommand{\VV}{\bm V} 
\newcommand{\R}{\mathbb R}
\newcommand{\ASM}{\mathrm{ASM}}
\newcommand{\RASM}{\mathrm{RASM}}
\newcommand{\bigO}{{\mathcal{O}}}
\newcommand{\abs}[1]{{\left\lvert #1 \right\rvert}}
\newcommand{\norm}[1]{{\left\lVert #1 \right\rVert}}
\newcommand{\tcolon}{{ : }}
\newcommand{\ip}[2]{{\left\langle #1, #2 \right\rangle}}

\usepackage{xspace}
\makeatletter
\DeclareRobustCommand\onedot{\futurelet\@let@token\@onedot}
\def\@onedot{\ifx\@let@token.\else.\null\fi\xspace}
\def\eg{{e.g}\onedot} \def\Eg{{E.g}\onedot}
\def\ie{{i.e}\onedot} \def\Ie{{I.e}\onedot}
\def\cf{{c.f}\onedot} \def\Cf{{C.f}\onedot}

\DeclareMathOperator{\sspan}{span}
\DeclareSIUnit\year{a}
\sisetup{retain-unity-mantissa = false}

% \authorrunninghead{BROWN ET AL.}
% \titlerunninghead{TME FOR HYDROSTATIC ICE SHEET FLOW}

\title{Achieving textbook multigrid efficiency for hydrostatic ice sheet flow}
\author{%
  {Jed Brown}\thanks{Versuchsanstalt f\"ur Wasserbau, Hydrologie und Glaziologie (VAW), ETH Z\"urich, 8092 Z\"urich, Switzerland (\email{brown@vaw.baug.ethz.ch})}
  \and {Barry Smith}\thanks{Mathematics and Computer Science Division, Argonne National Laboratory, Argonne, IL 60439, USA (\email{bsmith@mcs.anl.gov})}
  \and {Aron Ahmadia}\thanks{Supercomputing Laboratory, King Abdullah University of Science and Technology, Thuwal, Makkah, Saudi Arabia (\email{aron.ahmadia@kaust.edu.sa})}}
\begin{document}
\maketitle
\begin{abstract}
The hydrostatic equations for ice sheet flow offer improved fidelity compared to the shallow ice approximation and shallow stream approximation (SSA) popular in today's ice sheet models. Nevertheless, they present a serious bottleneck because they require the solution of a 3D nonlinear system, as opposed to the 2D system present in SSA.  This 3D system is posed on high-aspect domains with strong anisotropy and variation in coefficients, making it expensive to solve using current methods.  This paper presents a Newton-Krylov multigrid solver for the hydrostatic equations that demonstrates textbook multigrid efficiency (an order of magnitude reduction in residual per iteration and solution of the fine-level system at a small multiple of the cost of a residual evaluation).  Scalability on Blue Gene/P is demonstrated, and the method is compared to various algebraic methods that are in use or have been proposed as viable approaches.
\end{abstract}
\begin{keywords}
  hydrostatic, ice sheet, Newton-Krylov, multigrid, preconditioning
\end{keywords}
\pagestyle{myheadings}
\thispagestyle{plain}
\markboth{BROWN, SMITH, AND AHMADIA}{TEXTBOOK MULTIGRID EFFICIENCY FOR HYDROSTATIC ICE SHEET FLOW}

\input{tme-body}

\section*{Acknowledgments}
We are grateful to Edward L. Bueler for his helpful commentary on an early draft of this paper.
This work was supported by Swiss National Science Foundation Grant 200021-113503/1, U.S. Department of Energy's Office of Science Ice Sheet Initiative for CL-imate ExtremeS program under Contract DE-AC02-06CH11357, and the Shaheen Supercomputing Laboratory at KAUST.

\bibliographystyle{siam}
\bibliography{jedbib/jedbib}

\bigskip
% The text below can be removed in the published document
\begin{quotation}
The submitted manuscript has been created by UChicago Argonne, LLC,
Operator of Argonne National Laboratory (``Argonne'').  Argonne, a
U.S. Department of Energy Office of Science laboratory, is operated
under Contract No. DE-AC02-06CH11357.  The U.S. Government retains for
itself, and others acting on its behalf, a paid-up nonexclusive,
irrevocable worldwide license in said article to reproduce, prepare
derivative works, distribute copies to the public, and perform
publicly and display publicly, by or on behalf of the Government.
\end{quotation}
\end{document}
